% !TEX root = ../main.tex

\chapter{Blockchain (D. Marazzina)}

A blockchain is a \textbf{DL} (Distributed Ledger\footnote{in Italian: \textit{libro mastro}.}) where data are replicated, shared, and synchronized all over the world. \textbf{No one is the owner}. Any involved block cannot be altered retroactively, without the alteration of all the subsequent blocks. Blocks are connected in a chain.
\begin{itemize}
	\item \textbf{Trust} is not required in the blockchain;
	\item It is required in Dropbox, Google Drive, etc.
\end{itemize}

What happens if a node (by mistake or not) propagates an \textbf{incorrect transaction}? The system must not validate it. How? Through a \textbf{consensus mechanism}.

The problem of the \textbf{Byzantine generals} is a computer problem on how to reach consensus in situations where errors are possible. The problem consists of finding an agreement, communicating only through messages, between different components of conflicting information.

\fg{0.7}{55d25ddfe4b1fb3b630a4242136cf307-RJJ0KnWjoZ-image}

\section{Bitcoin}

Bitcoin is an electronic currency created in 2009 following the famouse paper\footnote{\url{https://bitcoin.org/bitcoin.pdf}} published in November 2008 by the inventor known under the pseudonym of \textbf{Satoshi Nakamoto}.

Keywords:
\begin{itemize}
	\item Consensus mechanism for lack of trust
	\item Decentralized database
	\item Cryptography
	\item Generation of new money
\end{itemize}

\section{Decentralization}

\begin{itemize}
	\item Politically decentralized: Bitcoin is not controlled by a specific entity.
	\item Architecturally decentralized (or distributed): Bitcoin lives on machines spread across the planet.
	\item Logically centralized: all machines that Bitcoin lives on agree on a shared status (e.g. how many Bitcoins each has).
\end{itemize}

A blockchain is a distributed (or architecturally decentralized) and politically decentralized database managed by a peer-to-peer network through distributed consensus mechanisms (to agree on a shared state, as logically centralized).\footnote{Visit this website to explore it! \url{https://www.blockchain.com/explorer}}
\begin{itemize}
	\item A \textbf{transaction} is a Bitcoin transfer that is included in the Blockchain. A transaction is identified by the two addresses (counterparts) and by the amount exchanged. The cryptographic private key signature of the two counterparts provides \textbf{proof of their identity} and that the Bitcoins come from the real owner. The signature also \textbf{prevents that the transaction (once executed) is altered} by others. The two addresses are publicly identified by a code.
\end{itemize}

\fg{0.7}{b5a948c1859f2826ea6b5ac0d3a83474-Pmg6G5phwe-image}

\begin{itemize}
	\item A \textbf{block} is a part of the Blockchain that contains a set of (Bitcoin) transactions. On average, a new block, which includes transactions, is added to the Blockchain every $10$ minutes (the system is tuned to make so happen).
	\item \textbf{Confirmation:} to be included in a block, and therefore in the Blockchain, a transaction must be confirmed by several nodes (\textbf{miners}), i.e. processed by the network.
	\item \textbf{Mining:} it is a mathematical process performed by the Blockchain nodes on the candidate block: once solved the block (and therefore all the transactions recorded in it) is confirmed and inserted into a block of the Blockchain. As a reward for their service, miners receive newly created Bitcoins (and the transaction fees). Mining $1$ new block now makes you earn $6.25$ BTC, mining can be done with ad hoc hardware and consumes a lot of energy.
	Miners accumulate transactions all the time (selecting them with a certain criterion, such as those that offer the highest fees). When they have enough transactions to fill a block (which has a fixed size) they try to solve the mining problem. The first miner that solves the mining problem propagates the block in the network, which is verified (they verify if the \textit{Proof of Work} is valid) by the other nodes and added to their copy of the blockchain.
\end{itemize}

\fg{0.7}{804e766ecde2775d6efff4a980e6a2ab-b0x52ZU7IL-image}

\begin{itemize}
	\item \textbf{Cryptographic signature}: it is a mathematical mechanism that allows you to prove the identity and ownership of Bitcoins in this case.
	When your Bitcoin software marks a transaction with your private key, the entire network can recognize you (not as an individual but as an id) and verify that the Bitcoins spent are in your possession. The transaction becomes public with respect to your code. There is currently no way for other nodes to guess your private key, to steal your Bitcoins.
\end{itemize}

\fg{0.7}{6133e4c47407f4c52e44ce591f19e7c7-bAzikty36C-anatomy_of_a_transaction}

\section{Asymmetric (or public key) encryption}

\begin{itemize}
	\item Step 1: a public-private key pair is generated for Alice
	\item Step 2: Bob uses Alice's public key to encrypt the message. Only those who have Alice's private key can decrypt it!
\end{itemize}

So, if Alice has kept the private key hidden, only she can read the message

\fg{0.7}{5094767bdd40d5f5696a43c1da84a0ad-1TwKZy3l5S-image}

Among the asymmetric cryptographies, Bitcoin uses \textbf{Elliptical Cryptography}.

Public keys are based on the creation of a mathematical problem that is very difficult to solve without information, but which - with the use of some information (the private key) - becomes easy and quick to solve. The user publicly distributes the problem (the public key) and keeps the additional information hidden (the private key).

\textit{An example: RSA encryption.}

Let's take two prime numbers, \texttt{13} and \texttt{7} (and don't reveal them to anyone).

Let's multiply them together: we get \texttt{n=91}.

We choose a public key: \texttt{5}.

We use the \textit{Extended Euclidean Algorithm} with inputs \texttt{5}, \texttt{7} and \texttt{13}, and we get the private key: \texttt{29}.

So \texttt{n=91}, public key \texttt{5}, private key \texttt{29}.

Let Alice and Bob know that \texttt{n=91}, Alice's public key is \texttt{5}, and her private key is \texttt{29}.

Bob wants to send a message to Alice: the message is \texttt{3}.

Then multiply \texttt{3} by itself \texttt{5} (public key) times, modulo \texttt{n=91}: \texttt{3*3*3*3*3=243}, \texttt{mod(243,91)=61}.

The public message is \texttt{61}.

By doing the opposite we can decode the message.

\begin{verbatim}
a=3; at=3;
for i=1:5-1
    at=mod(at*a,91)
end
Output=61.

a=61; at=61;
for i=1:29-1
    at=mod(at*a,91)
end
Output=3.
\end{verbatim}

We have decoded the message!

\begin{itemize}
	\item Alice and Bob know \texttt{n=91}.
	\item Everyone knows \texttt{5}.
	\item Only Alice knows \texttt{29}.
	\item Knowing \texttt{5} (public key) is not sufficient to decrypt the message.
	\item Can Bob trace Alice's private key? That is, is the information \texttt{91} sufficient to compromise security?
	\item Mathematical problem: factorization into prime numbers.
\end{itemize}

Prime factorization is not the most difficult problem on a bit-to-bit basis. Specialized algorithms such as Quadratic Sieve and General Number Field Sieve were created to address the factorization problem and have had some success. These algorithms are faster and less computationally intensive than the naive approach of guessing pairs of known primes.

These factoring algorithms become more efficient as the size of the numbers considered increases. The gap between the difficulty of factoring large numbers and multiplying large numbers narrows as the number (i.e. the key bit length) increases. As the resources available to decrypt the numbers increase, the size of the keys must grow even faster. This is not a sustainable situation for mobile and low-power devices that have limited computational power. The gap between factoring and multiplication is not sustainable in the long term.

All of this means that RSA is not the ideal system for the future of cryptography.

Techniques based on Elliptical Curves: \url{https://youtu.be/2RShPGAZqMs}.

Asymmetric algorithms guarantee confidentiality in communication. A message encrypted with the recipient's public key ensures that only the latter can decrypt this message, as it is the only one who has the corresponding private key.

Furthermore, by reversing the use of keys, that is, encrypting with the sender's private key and decrypting with the sender's public key, it is possible to guarantee authentication. It is on this principle that the digital signature is based.

The message is encrypted with the private key, so that anyone can, using the public key known by everyone, decrypt it and, in addition to being able to read it in clear text, be sure that the message was sent by the owner of the private key corresponding to the public one used to read it.

\section{What about miners?}

In addition to checking that the digital signatures of the transactions are authentic, that there is no double-spending, etc., and therefore verifying the transaction, their main job is to \textbf{build the Blockchain}, block by block. How? By concatenating the blocks together, this happens with the \textbf{solution of a computationally onerous mathematical problem}. If it were easily fixed, it would also be easy to \textbf{attack} the Blockchain.

\section{Proof of Work}

The Bitcoin Blockchain is therefore based on \textit{Proof of Work}.

Proof of Work, or PoW, is an algorithm that is used by different cryptocurrencies - such as Bitcoin and Ethereum - to reach a decentralized agreement between different nodes in the process of adding a specific block to the blockchain.

\textit{Hashcash (SHA-256)}\footnote{Visit this site to try some hashing: \url{https://emn178.github.io/online-tools/sha256.html}} is the Proof of Work feature used by Bitcoin. Cryptocurrency forces miners to solve extremely complex and computationally difficult mathematical problems to add blocks to the blockchain. This function produces a very specific type of data that is used to verify that a significant amount of work has been done.

A hash function takes an input of \textit{any} length and creates an output of \textit{fixed} length.
\begin{equation*}
	\texttt{a0680c04c4eb53884be77b4e10677f2b}
\end{equation*}
This is referred to as the \textit{message digest}.

It is also known as the digital fingerprint.

Every transaction has a hash associated with it. In a block, all of the transaction hashes in the block are themselves hashed (sometimes several times, the exact process is complex), and the result is the Merkle root. In other words, the Merkle root is the \textit{hash of all the hashes of all the transactions in the block}.

Our target is to \textbf{find a variation of it that SHA-256 hashes to a value beginning with `000'}. We vary the string by adding an integer value to the end called a \texttt{nonce} and incrementing it each time. Finding a match for ``Hello, world!'' takes us 4251 tries (but happens to have zeroes in the first four digits).

A PoW system \textbf{discourages attacks} and other abuse of service by enforcing certain jobs that require a computer's high processing time. A key feature of these works is their asymmetry: the work must be moderately, but easy to control.

Game theory: the nodes of the network compete with each other to register the new block thus obtaining compensation. Thanks to the expected remuneration, most of the nodes have an \textbf{interest in behaving in a \emph{legitimate} manner}, thus contributing to the growth of the blockchain.

\section{Proof of Stake}

Proof of Stake is an alternative method, a way by which nodes reach a consensus.

\fg[First proposal of the Proof of Stake.]{0.7}{pos}

In Proof of Stake you stake your money apart, and the greater the stake, the higher is the probability that you will be chosen to create a new block. Blocks are not mined, but minted. The idea is that you should not fraud the system, for example by adding fake transactions to the block, otherwise the money you put at stake will be taken from you. Provided you stake enough money, it should be more convenient to be fair. Moreover you are incentivized to do it because you get a reward.

The real problem is how to choose the validators, because rich people have more money to stake, thus more chances of being chosen as validators and becoming even richer.

There is no waste of energy, and there is no vulnerability to 50\%+1 attacks.

\section{Uniqueness}

The chain is unique: if two different chains are created (for example, due to a delay in updating the blocks), the \textit{incorrect} one is automatically deleted and overwritten by the correct one.

To determine the correct chain, there are mechanisms: the chain considered correct is always the \textbf{longest one} (longest chain rule). \textbf{The blockchain rewards computational effort.}

Generally, the longest chain is also the one stored by the majority of the network nodes (50\%+1 rule), given that \textbf{miners are constantly incentivized to create valid blocks} through the fees and Bitcoins generated at each resolution of the mining problem.

From a game theory perspective, collaborating on the longest chain is a balance, as no one can achieve a better result for themselves (without collaborating with others) by creating invalid blocks (PoWs). \textbf{This makes it very difficult for a hacker to change even the last block}.

In \textbf{March 2013}\footnote{\url{https://github.com/bitcoin/bips/blob/master/bip-0050.mediawiki}}\footnote{\url{https://freedom-to-tinker.com/2015/07/28/analyzing-the-2013-bitcoin-fork-centralized-decision-making-saved-the-day/}}\footnote{\url{https://twitter.com/gavinandresen/status/311290936527298561}} there was a problem, because the Bitcoin software was being updated from version 0.7 to 0.8. It happened that a new block was created by a miner who had version 0.8, and was correctly recognized to miners who had 0.8, but those with version 0.7 rejected it.

In the end miners decided to switch back to 0.7, since ``\emph{we cannot get every bitcoin user in the world to now instantly switch to 0.8 so no, we need to rollback to the 0.7 chain}'' (Pieter Wuille). Eventually 24 blocks would be lost.

\fg[\texttt{BTC Guild} was a key player in solving the March 2013 fork, having at disposal 20-30\% of the hashpower.]{0.9}{c5d0d8444570c81ebcd5ac910f0741e7-HutyWMPv53-image}

\section{How the nodes get paid}

Who pays the nodes? \textbf{The system itself}. When a new block is added, besides the fees that people paid with the transactions inside it to incentivize the miner to add them to the block, the miner also receives a quantity from the system, meaning that \textit{everyone agrees} that they made enough computational effort. In other words, if everyone in the world suddenly decides to agree that you have a million dollars, even if you physically don't, you can go buy a Lamborghini, and then your fictitious money gets transferred to the company. It looks strange, but if you think about it, it's the process of printing virtual money for central governments to make inflation happen.

\fg{0.7}{d234ce63f4f4201295cf60ebe5083522-mZ9rNv3xhE-image}

\section{How much does it cost to produce a Bitcoin?}

Until a few years ago a good computer was enough to produce (mine) Bitcoin, today thousands of very powerful processors and graphics cards are needed.

For the extraction of Bitcoins, computers with specific processors called ASICs (application-specific integrated circuits) have been developed; these processors, in addition to having a high cost, also involve a \textbf{high consumption of electricity}.

It's \textit{not real-time} technology, it may take up to 1-2 hours to complete a transaction.

\section{Notarization}

Notarization is a very useful \textbf{application} of the Blockchain. Suppose you want to prove that a certain document (a thesis for example) was ready on a given day.

Suppose the hash of the file is recorded in the blockchain. All parties having access to the blockchain can individually verify the authenticity of these files.

It is not possible to attach documents to the blockchain. The procedure is the following:
\begin{enumerate}
	\item The document is hashed. This means that the content of the document is \textit{summarized}, or better represented, by a string of 64 characters (256bit).
	\item The 64-character string is then put into a Bitcoin (or other) blockchain transaction using the \texttt{OP\_RETURN} \textbf{field}, it's a \textit{free} field presented in every Bitcoin transaction: free, but limited in size.
	\item A small Bitcoin payment is made to process the transaction and \textit{register it on the blockchain} (\textbf{you can also make a 0 BTC transaction}, so paying only the cost of the transaction - variable but around $0.00001$ BTC).
\end{enumerate}

Hashing is \textit{irreversible} once you know the hash, it is \textit{mathematically impossible} to deduce the original text.

\textbf{Is it possible that two different files report the same Hash?}

The hash projects a (theoretically infinite) sequence of data, in a finite hash and is usually much shorter than the original file, so duplicates can happen. Possible, but \textit{very unlikely}! And, with the most advanced hashing techniques, it is difficult (if not impossible) to use this vulnerability (known as \textbf{Collision}) to modify documents for illegal purposes. And then there is the avalanche effect, a property where a small variation in input produces a considerable variation in the hash.

\section{Scalability}

It is not yet possible to think of a DLT that \textbf{completely replaces the system of financial transactions} (too large-scale). But it would be possible for two counterparties to create a DLT and use it for their transactions.

The cryptography used in Blockchain could certainly be considered to make current databases more secure.

Not just finance: The Blockchain can be used as a register in which to enter any type of information and consequently also a contract, an act or a certificate, avoiding the intermediation of third parties, but maintaining the guarantee of advertising.

Comparison of transactions capacity\footnote{\url{https://altcointoday.com/Bitcoin-ethereum-vs-visa-paypal-transactions-per-second/}}
\begin{itemize}
	\item Bitcoin – 3 to 4 transactions per second (1 block every 10 minutes);
	\item Ethereum – 20 transactions per second (1 block every 15 seconds);
	\item PayPal – 193 transactions per second average;
	\item Visa – 1.667 transaction per second.
\end{itemize}

PoW is \textbf{expensive:} it is the main cause of scalability problems (as well as memory problems: all nodes should store all transactions).

Solutions under study: off-chain operations, Proof of Stake, Proof of Work only on some random blocks, permissioned ledgers.

Permissioned ledgers (or private blockchains) can be controlled, and therefore can have ownership. When a new data or record is added to the blockchain, the approval system is not bound to the majority of participants but to a limited number of actors who can be defined as trusted.

\textbf{Three false myths?}
\begin{enumerate}
	\item Less expensive system
	      \begin{enumerate}
	      	\item Bitcoin transactions on the Blockchain cost (nearly) nothing.
	      	\item But this is possible because new Bitcoins are created for the remuneration of the network nodes that work for the proper functioning of the system.
	      \end{enumerate}
	\item Safer
	      \begin{enumerate}
	      	\item Potentially yes: safer in terms of immutability (less than 51\% attacks), but not privacy.
	      	\item The system of remuneration of the nodes pushes to use their own computing power to make the system work, not to try to force it.
	      	\item The database is distributed all over the world.
	      \end{enumerate}
	\item Faster
	      \begin{enumerate}
	      	\item The current financial transaction system is \textit{slow} not for a technological issue, but for a consensus issue.
	      	\item ``Consensus by reconciliation'': a process that the financial markets have chosen as their ``checks and balances'' system.
	      	\item It is, therefore, a question of moving to a new form of ``decentralized consensus'': an automatic mechanism that allows transactions to be quickly validated. Like in Bitcoin!
	      \end{enumerate}
\end{enumerate}

Read here to see how to verify a block on the blockchain!

{\scriptsize\url{https://blockchain-academy.hs-mittweida.de/courses/blockchain-introduction-technical-beginner-to-intermediate/lessons/lesson-13-Bitcoin-block-hash-verification/topic/how-to-calculate-and-verify-a-hash-of-a-block/}}

\section{Cryptoassets (D. Marazzina)}

Every coin that is not Bitcoin is referred to as \textbf{altcoin}.

A \textbf{cryptoasset} (coin) is a financial asset that is created and exchanged on a platform; like a blockchain, but not only a blockchain.

Assets are divided into
\begin{itemize}
	\item \textbf{cryptocurrencies:} the native currency of the blockchain (Bitcoin)
	\item \textbf{tokens:} crypto asset not native of the technology
\end{itemize}

Is Bitcoin a token? Not exactly. While all cryptocurrencies are technically tokens, Bitcoin is usually not considered a token, but a coin. This is because it is generated by the blockchain that supports it. Tokens, on the other hand, operate on existing blockchains. In fact, a single blockchain offers space for \textbf{many different tokens}. Suffice it to say that the Ethereum network currently hosts almost half a million.

It's not easy to \textit{diversify} cryptoassets because correlation is very high.

It may happen that there are so called \textbf{forks} on the Blockchain:
\begin{itemize}
	\item \textbf{Soft fork:} when there are two versions of the Blockchain and it's solved automatically.
	\item \textbf{Hard fork:} when something (a software update that only some install) creates two versions of the Blockchain.
\end{itemize}

\paragraph{Polkadot} multi-chain etherogeneous and scalable technology that gives freedom on the structure and the chains of its network (Ethereum is stricter). Consensus is granted on all the chains, which all communicates.\footnote{\url{https://en.wikipedia.org/wiki/Polkadot_(cryptocurrency)}}
\textbf{DOT} is the cryptocurrency of the Polkadot platform.

\paragraph{Ripple} Built for enterprise use, Ripple aims to be a fast (4 seconds), cost-efficient cryptocurrency for cross-border payments. Within the Ripple system, money isn't actually transferred from one place to another: only the promise of payment is transferred.
It's designed to find the most convenient way to exchange two cryptocurrencies. The transaction fee is paid and then destroyed by the network.\footnote{\url{https://en.wikipedia.org/wiki/Ripple_(payment_protocol)}}
\textbf{XRP} is the cryptocurrency used by the Ripple payment network.




\paragraph{Bitcoin Cash.}

It's a cryptocurrency that is a fork of Bitcoin that was created in 2017:
\begin{itemize}
	\item On 21 July 2017 Bitcoin miners locked in a software upgrade referred to as \textbf{Bitcoin Improvement Proposal (BIP) 91: the Segregated Witness (Consensus layer).}
	\item SegWit alleviates the scaling problem by enabling the Lightning Network, an overlay network of micropayment channels, hypothetically resolving the scaling problem by enabling virtually unlimited numbers of instant, low-fee transactions to occur \textit{off chain}.
	\item By 8 August 100\% of the Bitcoin mining pools signaled support for SegWit.
	\item A small group of mostly China-based Bitcoin miners, that were unhappy with Bitcoin's proposed SegWit improvement plans, pushed forward alternative plans for a split which \textbf{created Bitcoin Cash}. The proposed split included a plan to increase the number of transactions its ledger can process by increasing the block size limit.
\end{itemize}

In 2018 Bitcoin Cash subsequently split into two cryptocurrencies: Bitcoin Cash, and Bitcoin SV.

\paragraph*{Bitcoin SV.}
\begin{itemize}
	\item On 15 November 2018, a hard fork chain split of Bitcoin Cash occurred between two rival factions called Bitcoin Cash and Bitcoin SV.
	\item The split originated from what was described as a \textit{civil war} in two competing Bitcoin cash camps.
	\item The first camp promoted the software entitled Bitcoin ABC (short for Adjustable Blocksize Cap) which would maintain the block size at 32 MB.
	\item The second camp put forth a competing software version Bitcoin SV, short for \emph{Bitcoin Satoshi Vision}, that would increase the block size limit to 128 MB.
\end{itemize}

\paragraph*{Bitcoin Gold.}
Another hard fork of Bitcoin.
\begin{itemize}
	\item Bitcoin Gold is a fork of Bitcoin that seeks to reduce the influence of miners who use specialized equipment known as ASICs. The team's stated goal is to ``make Bitcoin decentralized again.''
	\item Bitcoin Gold hard forked from the Bitcoin blockchain on October 24, 2017.
	\item In May 2018, Bitcoin Gold was hit by a 51\% hashing attack by an unknown actor. This type of attack makes it possible to manipulate the blockchain ledger on which transactions are recorded and to spend the same digital coins more than once. During the attack, 388,000 BTG (worth approximately \$18M) was stolen from several cryptocurrency exchanges.
	\item Bitcoin Gold suffered from 51\% attacks again in January 2020\footnote{{\scriptsize\url{https://thenextweb.com/hardfork/2020/01/27/Bitcoin-gold-51-percent-attack-blockchain-reorg-cryptocurrency-binance-exchange/}}}.
\end{itemize}

\paragraph{TheDAO incident}\label{dao-incident}\footnote{\url{https://medium.com/swlh/the-story-of-the-dao-its-history-and-consequences-71e6a8a551ee}}

The Decentralized Autonomous Organization (known as TheDAO) was meant to operate like a venture capital fund for the crypto and decentralized space. The lack of a centralized authority reduced costs and in theory provides more control and access to the investors.

On \textbf{June 17, 2016}, a hacker found a loophole in the coding that allowed him to drain funds from TheDAO. In the first few hours of the attack, 3.6M ETH were stolen, the equivalent of \$70M at the time.

It’s important to understand that this bug did not come from Ethereum itself, but from this one application that was built on Ethereum.

If the blockchain is immutable, how can it be canceled?

Two solutions:
\begin{itemize}
	\item \textbf{Soft Fork}: doing a software update to lock the funds (but would remain on the attackers wallet).
	\item \textbf{Hard Fork}: go back to a previous version of the blockchain.
\end{itemize}

A hard fork was made, and the blockchain was rewritten by eliminating the transaction.

How? With distributed consent! More than 50\% of Ethereum users have accepted the update. The new separate version became \textbf{Ethereum (ETH)} with the theft reversed, and the original continued as \textbf{Ethereum Classic (ETC)}.

\textbf{Distributed consent overcomes the immutability of the Blockchain.}

\section{Derivatives}

A \textbf{futures contract} is an agreement between counter-parties to buy or sell an asset at an explicit price and date in the future. The buyer is obligated to buy the underlying asset a specific price once the contract expires, and the seller is required to furnish the asset at the time of expiry.

In \textbf{inverse futures contract} BTC are the coin while USD are the commodity.

\paragraph{BTC Perpetual Swap} Theese are perpetual inverse futures contract. \emph{Perpetual} means without expiration. For example I give you 20 BTC and you give me 40 USD, I'm short on BTC; if BTC goes down you're losing -- paying me. Every 8 hours we repeat. It's a way of \textbf{funding}, the longs will pay and the shorts will receive.

\section{Decentralized Finance (DeFi)}

\textbf{DeFi} is an ambitious attempt to decentralize core traditional financial use cases like trading, lending, investment, wealth management, payment and insurance on the blockchain. Good trust on Ethereum, even though usability is not awesome.

\paragraph{Lightning Network} Layer 2 payment protocol on top of a blockchain cryptocurrecy where payments can happen fast solving the scalability problem. The idea is: you open a \emph{payment channel} by queueing it in the blockchain, then you make a lot of fast transactions, finally you close the channel and only the final version of the transactions in broadcasted.

\section{Fungible and Non-fungible Token (NFT)}

\fg{1}{nft}

\paragraph{CryptoKitties} is a blockchain game on Ethereum developed by Axiom Zen that allows players to purchase, collect, breed and sell virtual cats. It is one of the earliest attempts to deploy blockchain technology for recreation and leisure. The game's popularity in December 2017 congested the Ethereum network, causing it to reach an all-time high in number of transactions and slowing it down significantly.

CryptoKitties operates on Ethereum's underlying blockchain network, as a \textbf{non-fungible token}, unique to each CryptoKitty.

Several traits can be passed down from the parents to the offspring. There are a total of 12 \emph{cattributes} for any cat, including pattern, mouth shape, fur, eye shape. Other features like cool down times are not passed down but are instead a function of the \emph{generation} of the offspring, which is one higher than the maximum generation between the two parents. In December 2017 a CryptoKitty sold for \$100.000.

A CryptoKitty's ownership is tracked via a smart contract on the Ethereum blockchain.

\paragraph{Ethereum's Tokens} Any \textbf{fungible} commodity can be represented by a token derived from \textbf{ERC-20 standard}. \textbf{NFTs} can be derived from \textbf{ERC-721 standard}. This recent standard has made possible to represent goods like pieces of art, personalized access rights, properties.

\begin{itemize}
	\item More than 80\% of DeFi solution is on Ethereum
	\item Algorand is a promising technology (but young)
\end{itemize}

\fg[Classification of tokens.]{0.8}{token-classification}

\fg[Example of a restaurant voucher.]{0.8}{token-classification-restaurant}

\section{Coin vs Token}

\paragraph{Coin} It's a unit of value that is native to a blockchain. It is a means of exchange within the blockchain to incentivize the network of participants to use the blockchain. The cryptocurrencies Bitcoin, Ether, Ripple, and Litecoin are all examples of native cryptocurrencies. The sole purpose of a cryptocurrency is for exchange of value, and it has limited functionality beyond that.

\paragraph{Token} It's a piece of business logic (i.e., ``smart contract'') coded into an existing blockchain. A token can have a functionality beyond an exchange of value - it can represent any asset or functionality desired by the developer for use on a platform. Most tokens are created on the Ethereum blockchain by using ERC-20 smart contracts, such as Tron and VeChain.

Coins and tokens differ not only for the IT definition, but also for \textbf{market behaviour}.

\section{How to create a crypto?}

Tokens are often released through a crowdsale known as an initial coin offering (ICO) in exchange for existing coins, which in turn fund projects like gaming platforms or digital wallets. You can still get publicly available tokens after an ICO has ended-similar to buying coins-using the underlying currency to make the purchase.

\begin{enumerate}
	\item \textbf{Build Your Own Blockchain} or \textbf{Fork an Existing One}. Both methods require quite a bit of technical knowledge. You can fork an existing blockchain by taking the open-source code found on Github making a few changes, and launching a new blockchain with a new name.
	\item Launch a Coin or Token Using a Cryptocurrency Creation Platform -- all you have to do is to choose a platform, pay for this service, and then enter the parameters, from the logo to the number of coins awarded for signing a block.
	This is faster, simpler, and cheaper than creating a coin because it doesn't require the time and effort to build and maintain a new or forked blockchain and instead relies on the technology already in use for Bitcoin or Ethereum.
\end{enumerate}

Then you need to find someone which will buy the coin or token. This is not easy: you need a community interested in your coin/token: usually it is easier if the coin/token is connected to a real project.

Plenty of cryptocurrencies are unsuccessful, even questionable from a legal standpoint, because the ICO wasn't created in good faith or the coin failed to generate lasting interest.

\section{Exchange}

You can visit \url{https://blog.coingecko.com/trust-score/} to see the best exchanges according to some score.

\paragraph{Quadriga} Gerald Cotten created Quadriga, a Canadian exchange, where he controlled fake accounts with fake money. People exchanged real money with these fake accounts and were scammed for \$135M. See\\
\url{https://www.youtube.com/watch?v=NMDZcbkgpJA}\\
\url{https://www.youtube.com/watch?v=jT55oLyT0ac}

\paragraph{Bitgrail} Nearly one year after Francesco Firano, the owner and operator of the Bitgrail cryptocurrency exchange, announced that the exchange had lost 17M NANO (approximately \$170M), an Italian Bankruptcy Court and a court-appointed technical expert concluded that Mr. Firano (``The Bomber'' as Mr. Firano called himself on social media) was at fault for the loss and is required to return as much of the assets to his customers as possible. The ruling is a landmark decision that sets an important precedent for the protection of cryptocurrency users worldwide. In its decision, the court concluded that both Bitgrail and Mr. Firano, personally, be declared bankrupt, authorizing seizures of many of Mr. Firano’s personal assets. So far, authorities have seized over \$1M in personal assets, including Mr. Firano’s car. Millions of dollars in cryptocurrency assets have been seized from Bitgrail’s exchange accounts and moved to accounts managed by trustees appointed by the Court. The Court found that the NANO reported lost by Mr. Firano on February 9, 2018, had actually been removed from the exchange months earlier, between July 2017 and December 2017. The Court criticized Mr. Firano for not immediately taking steps to account for the losses.

\section{Smart Contracts}

Ethereum
\begin{itemize}
	\item It is another platform developed on blockchain technology.
	\item In order to be able to run on the peer-to-peer network, Ethereum users pay for the use through a cryptocurrency, called Ether, which therefore acts both as a cryptocurrency and as a fuel.
	\item Ethereum is different from the Bitcoin blockchain since \textbf{it is constructed to create Smart Contracts}. Even in Bitcoin, elementary smart contracts can be created. In Ethereum the peculiarity is that they are \textbf{listed on a Turing-complete language}.
	\item A smart contract is a program that runs on Ethereum's decentralized virtual machine.
	\item The smart contract cannot act actively (for example, it cannot monitor the status of another smart contract and act accordingly), but it acts reactively (a user, not necessarily directly, must invoke it to perform any operation).
	\item The smart contract cannot interrogate the world outside the blockchain. For example, in the case of a sports bet managed through a smart contract, this cannot obtain the result of the game from a website and act accordingly.
	\item The smart contract can instead obtain information using \textbf{oracles}.
	\item Oracles are smart contracts in which information from the outside world is loaded in exchange for a reward.
\end{itemize}

\emph{Example.} AXA is testing insurance on late flights using smart contracts.

\paragraph{Collateral} Put simply, collateral is an item of value that a lender can seize from a borrower if he or she fails to repay a loan according to the agreed terms. One common example is when you take out a mortgage. Normally, the bank will ask you to provide your home as collateral.
Collateral acts as a \textbf{guarantee} that the lender will receive back the amount lent even if the borrower does not repay the loan as agreed.

\emph{Example.} I want to create a tourism insurance contract against rain. Bob signed a tourism insurance contract for an holiday in New York for May 2, 2020. This contract will reimburse the payment for his trip (\$200) if that day it will rain. In the insurance contract there is this clause which defines the meaning of ``it will rain''. Rainfall amount is described as the \emph{depth of water reaching the ground}, typically in inches or millimeters (25 mm equals one inch). It rains more than 0.1 inch on that day, according to the Laguardia Airport Station result, reported by the website \url{https://www.wunderground.com/history/monthly/us/ny/new-york-city/KLGA}.

We need an \textbf{oracle}.
\begin{enumerate}
	\item The insurer creates the code of the smart contract.
	\item The insurer pays a third part to be an oracle, i.e. to report the website information to the smart contract.
	\item The insurer is ready to sell the insurance contract.
\end{enumerate}

Why Bob should prefer a blockchain-based insurance? Based on a \textbf{third part}, which has no reason to say that it did not rain, if it happened.
\textbf{Automatic payment}: if the oracle tells to the smart contract that it happened, the smart contract pays.

What happens if Bob wants to be paid in fiat money?

The smart contract can only say ``this contract is to be paid'', for example creating a null transaction with this sentence in the \texttt{OP\_RETURN} field. This is written on the blockchain, which is immutable. Then, the insurer, once he/she reads this on the blockchain can tell to his/her bank ``please, pay''.
So, not a real automatic payment without cryptocurrencies.

Possible solution: Stablecoins.

\section{Coin Offerings}

\subsection{Initial Coin Offering (ICO)}

\paragraph{Origins of the ICO} J. R. Willett needed to fund his idea (MasterCoin, a \emph{supplementary protocol to Bitcoin} with built-in support for custom tokens), so he asked for some BTC donations. He also added several features inside the protocol that would only be available to those owning MasterCoins.
Willet introduced the \textbf{first ICO}, and the first utility token.
Total raised: \$600K.

The MasterCoin protocol turned out to be incredibly successful. Rebranded to and known today as \textbf{Omni Layer (OMNI)}, it now serves as the underlying protocol for the \textbf{Tether token (USDT)}.

\textbf{NextCoin (NXT)} is a peculiar second ICO.

\paragraph{Ethereum}
In 2014 Vitalik Buterin tries raising funds for a new and noteworthy ICO for what he envisions the world's first zero-infrastructure platform, named \textbf{Ethereum}.
The token sale raised 3700 BTC in the first 12 hours.
Total raised: \$18.3M

Since Ethereum's inception, over 1.000 \textbf{ERC20} cryptocurrencies have been issued on the platform since its initial ICO. ERC20 itself is a protocol standard that defines certain rules and standards for issuing tokens on Ethereum's network and infrastructure. ERC stands for \emph{Ethereum Request For Comments} and \emph{20} stands for a unique ID number to distinguish this standard from others.

In 2016 a decentralized autonomous organization called TheDAO, a set of smart contracts developed on the platform, raised a record \$150M in a crowd/token sale to fund the project. \textbf{TheDAO was exploited} in June when \$50M in Ether were taken by an unknown hacker. Subsequently, Ethereum was split into two separate blockchains (see \ref{dao-incident}) -- the new separate version became \textbf{Ethereum (ETH)} with the theft reversed, and the original continued as \textbf{Ethereum Classic (ETC)}.
\textbf{Solidity}, initially proposed by Gavin Wood in August 2014, is the primary \textbf{programming language} (contract-oriented) used on the Ethereum platform.

\paragraph{Fall of ICO}
According to the ICORating, in the second half of 2018, the profitability of investments in blockchain startups decreased by 22\%, and 58\% of ICO projects announced in Q4 2018 were not able to raise more than \$100.000. Moreover, in Q4 2018 40\% of projects with previously announced ICOs have already deleted their social network accounts and websites

Cayman Island and British Virgin Islands rank among top ICO countries volume-wise.

\paragraph{Disadvantages and limitations}
The biggest disadvantage of an ICO investment is the \textbf{risk}. The market is volatile, and you never know the actual intentions of a newly minted company. The first risk can be called \textbf{ordinary fraud} when the team of the project pursues only one goal: to collect investors' money.
In addition, since \textbf{there are no laws at the moment} that would regulate the conduct of cryptocurrency crowdsales from the position of an investor, it cannot be ruled out that the project may not live to the stage of product appearance or disappoint the investor with its implementation.
Based on the statistical research provided by Satis Group, premier ICO advisory company, approximately 81\% of ICO's are \textbf{scams}, 6\% Failed, 5\% had \textbf{gone dead}, and 8\% went on to trade on an exchange.

Actually, one of the main reasons for such statistics might be \textbf{lack of token holders' control over their investments}, the absence of bills and laws regulating the legal field in the sphere of ICO.

The other serious threat is \textbf{hackers' attacks}. Research, conducted by Ernst\&Young (2017), showed that more than 10\% of all funds raised by ICO was stolen by cybercriminals. Analysts examined 372 ICOs conducted between 2015 and 2017. The monthly amount of losses from hackers in an ICO is \$1.5M. Moreover, \textbf{attackers often manage to gain access to personal data from investors}: from their addresses and phone numbers to billing information.

Before launching ICO, the development team specifies tasks, for which funds should be raised, and indicates in its whitepaper 2 figures: the minimum and maximum, which are called \textbf{Soft Cap} and \textbf{Hard Cap}.

\begin{itemize}
	\item Hard Cap -- defines the final goal, the upper limit of the amount of money invested, the most desired result. This is a very important indicator, precisely, because \textbf{many cryptocurrencies have a limit on the total number of units in circulation}. This, in turn, is one of the most important factors influencing the value of the coin, naturally, in addition to supply and demand.
	\item Soft Cap -- the minimum required amount of the investments for the team to proceed the project implementation in accordance with the plans. If it is not reached within the specified period, the contract is closed and automatically returns all funds raised to depositors. If Hard Cap is achieved, the sale of tokens stops. However, after overcoming soft cap investors control only purchased tokens and can't monitor invested money or withdraw part of the investment.
\end{itemize}

Finally, the big concern is the \textbf{connection between the token holders and holding company} and several relevant questions arise, for example, \textbf{what happens if the company which issued the tokens is sold or will the token holders have any rights under the new management}?

\subsection{Decentralized Autonomous Initial Coin Offering (DAICO)}

DAICO is a new fundraising model. Founder of Ethereum blockchain Vitalik Buterin proposed this model, combining the advantages of Decentralized Autonomous Organizations (DAO) with the classic ICO. This synergistic model allows you to make the process of collecting and spending funds as \textbf{transparent} and \textbf{safe} as possible.

DAICO is \textbf{based on a smart contract that regulates all actions} to attract and work with funds.

The difference between DAICO and ICO begins after the first stage when a mechanism called a \textbf{tap} is launched. Tap allows tokens holders to \textbf{control how much money is available to the team}. The \textbf{crane} determines the \textbf{amount per second that the development team can withdraw} from the contract. Payments to developers are made not once, but \textbf{gradually}, for example, once a month. If they need a larger amount than the one that is written in the smart contract, then this question is put to the \textbf{vote}. And holders of tokens can either approve this proposal or not.

\subsection{Initial Exchange Offerings (IEO)}

The cryptocurrency exchange is directly involved in the selection of projects, the organization of the tokensale and the sale of tokens.

There are several advantages of IEO which are in overcoming the disadvantages of ICO:
\begin{itemize}
\item \textbf{The risk of Scams for investors is lower}. The project is launched at the exchange after the serious procedure of verification. The exchange rejects the doubtful project in order to save its' reputation.
\item The process of \textbf{listing new tokens is faster}.
\item \textbf{Redistribution of costs becomes available}. According to Autonomous Research, there is \$1-3 M cost to list the token at the exchange. An IEO project has lower costs for listing.
\item The \textbf{speed of funding is higher}. In ICO the primary distribution of tokens may been lasted for several days whereas in IEO it lasts \textbf{several minutes} or even seconds.
\item Investors' gain is higher. The listing \textbf{token values are bigger} than in primary distribution.
\item There is \textbf{no need to open another wallet}.
\item The process of investing is simple, the investors need to replenish the balance on the exchange, wait for the token-sale and put an order for buying.
\item \textbf{Tokens are traded at an equal price}. This reduces the probability of falling rates from the early investors who purchased first.
\end{itemize}

At the same all these advantages have some drawbacks. For example, due to the high distribution speed, \textbf{some investors have no time to make an order and buy tokens of big projects}. Moreover, nowadays there is limited number of IEO, and they are not mainstream way of funding the projects. Reason for lower popularity is the \textbf{unwillingness of exchanges to take additional work}. The mentioned verification procedure is very strict, e.g. there is an obligation for \textbf{verification of identity}.

\emph{Example.} BitTorrent token (BTT) on Binance exchange\footnote{\url{https://icoholder.com/en/bittorrent-28385}} and LEO token on the Bitfinex exchange.

The evident advantage of IEO over an ICO is the presence of existing user base on the exchange platform that allows to raise tremendous investments even on the private sale stage.

\subsection{Security Token Offerings (STO)}

The popular utility tokens used in ICO have the main disadvantage, which is the \textbf{absence of any compensation for investors in case of ICO fail} as utility tokens are not security papers leading to the absence of any obligations for making beneficial conditions for investors.
\textbf{Security tokens represent real capital} in the enterprise, at the same time such token is \textbf{not necessarily tied to a share in the company}, it can be used to \textbf{divide the rights of the ownership}. In fact, they may \textbf{give the holder a number of rights}: ownership of shares, periodic dividends, cashflows, payment of debts, the right to vote, etc. All these rights are secured by a smart contract. Due to the features of these tokens, their value is supported by securities, therefore, they are considered an investment.
The issue of security tokens requires \textbf{serious supervision by regulatory authorities}. This supervision leads to the protection of investments and gives investors more rights, thus restoring the balance of power from the point of view of stakeholders.
\textbf{Security token offering (STO)} is the initial offer of security tokens. STO is similar to ICO in one thing: both issues token for investors. The main reason for buying securities token is \textbf{dividends or voting rights} .
STO project meets all the requirements of the SEC: investor's money is protected by law. In case of disputes, the investor can file a complaint with the appropriate authority since this type of token falls under the legislation on securities.

By 20th March 2019, 122 STOs have already been completed, raising \$512M.
54 Security Token Offerings are currently listed and ongoing.
Out of 328 STOs launched so far, only 12 of them have failed (3.65\%). STO raised \$1258 mln in overall.

\emph{Examples.} Bolton Coin, UniCrypy, GG World Lottery.

\fg{0.6}{1285a6b75181acc02dbcae61b217a6bc-0Z4VUukkZi-image}

\section{Stablecoins}

A stablecoin is a new class of cryptocurrencies that attempts to offer \textbf{price stability} and are \textbf{backed by a reserve asset}.

Stablecoins may be pegged to a currency like the U.S. dollar or to a commodity's price such as gold.

Stablecoins achieve their price stability via collateralization (backing) or through algorithmic mechanisms of buying and selling the reference asset or its derivatives.

Stablecoins have gained traction as they attempt to offer the best of both world's-the instant processing and security or privacy of payments of cryptocurrencies, and the volatility-free stable valuations of fiat currencies.

\emph{Example.} The Reserve protocol comprises two tokens: the Reserve token (RSV - a decentralized stablecoin) and the Reserve Rights token (RSR - a cryptocurrency used to facilitate the stability of the Reserve token and confers the cryptographic right to purchase excess Reserve tokens as the network grows).

\emph{Example.} bitCNY Pegged to the Chinese Yuan

\fg{0.8}{stablecoin_list}

\paragraph{Gold-Backed Stablecoins}
Usually one token of stablecoin equals one gram of the gold. Since they're tied to the gold, this stablecoin's price can't fall below the current price of the gold. A third party holds the gold in reserve.

\paragraph{LIBRA}

Morgan Beller planted the first seeds of Libra,\footnote{\url{https://sls.gmu.edu/pfrt/wp-content/uploads/sites/54/2020/02/LibraWhitePaper_en_US-Rev0723.pdf
}} she is known as the key figure behind Facebook's push into blockchain technology; she also considers herself a co-creator of Libra.\footnote{\url{https://www.publish0x.com/crypto-timelines/a-brief-timeline-of-libra-xqqdww}}

\begin{itemize}
\item May 2017: Morgan Beller left her job at Medium.com and started working at Facebook. Nine months later, she joined the blockchain strategy team and started working on Facebook blockchain initiative.
\item 8 May 2018: David Marcus (Facebook vice president) joined the new Libra team and officially began working on the new project. They recruited other top Facebook talent onto the new blockchain division.
\item 13 June 2019: The Wall Street Journal reported that Facebook is working on its own cryptocurrency that will be announced next week, and set to be launched next year alongside the blockchain-based network that will support it. The Wall Street Journal also reported that Facebook has secured backing from more than a dozen companies like PayPal, Visa, Mastercard, Stripe, Booking.com, and each company will invest around \$10M to fund the new currency development.
\item 18 June 2019: Facebook announced Libra digital currency, Calibra digital wallet, and the nonprofit Libra Association, an organization based in Geneva, Switzerland, that will govern the new currency.
\item 16 July 2019: David Marcus told the Senate Banking Committee in U.S. Senate Hearing, "that Facebook will only build its own Calibra cryptocurrency wallet into Messenger and WhatsApp, and will refuse to embed competing wallets" and also expressed that "Libra will not launch until the U.S. lawmakers' concerns have been answered".
\item 4 October 2019: PayPal withdraws from the Libra Association.
\item 9 October 2019: U.S. Senator Sherrod Brown and Senator Brian Schatz sent letters to the CEOs of Stripe, Visa, and MasterCard to express deep concerns over Facebook's Libra Association
\item 11 October 2019: eBay, Mastercard, Strip, Visa, and Mercado Pago withdraw from the Libra Association.
\item 15 October 2019: Facebook launches Libra Association. Representatives from the remaining 21 organizations (out of 28 original members) met and signed Libra Association charter in Geneva. 21 initial members include: Coinbase, Lyft, Uber Technologies, Spotify AB, PayU, Vodafone, Women's World Banking, Mercy Corps, Creative Destruction Lab any other organizations
\end{itemize}

\paragraph{Tether}\footnote{\url{https://tether.to/en/}}

Abstract. A digital token backed by fiat currency provides individuals and organizations with a robust and decentralized method of exchanging value while using a familiar accounting unit. The innovation of blockchains is an auditable and cryptographically secured global ledger. Asset-backed token issuers and other market participants can take advantage of blockchain technology, along with embedded consensus systems, to transact in familiar, less volatile currencies and assets. In order to maintain accountability and to ensure stability in exchange price, we propose a method to \textbf{maintain a one-to-one reserve ratio between a cryptocurrency token, called tethers, and its associated real-world asset, fiat currency}. This method uses the Bitcoin blockchain, Proof of Reserves, and other audit methods to prove that issued tokens are fully backed and reserved at all times.

Beginning with a whitepaper published online in January 2012, J.R. Willett described the possibility of building new currencies on top of the Bitcoin Protocol.
The precursor to Tether, originally named ``Realcoin'', was announced in July 2014. The first tokens were issued on 6 October 2014, on the Bitcoin blockchain.
On 20 November 2014, Tether CEO Reeve Collins announced the project was being renamed to ``Tether''.
In January 2015, the cryptocurrency exchange Bitfinex enabled trading of Tether on their platform.
In early 2018 Tether accounted for about 10\% of the trading volume of Bitcoin, but during the summer of 2018 it accounted for up to 80\% of Bitcoin volume.
It formerly claimed that each token was backed by one United States dollar, but on 14 March 2019 changed the backing to include loans to affiliate companies
On 30 April 2019 Tether Limited's lawyer claimed that each Tether was backed by only \$0.74 in cash and cash equivalent.

\paragraph{MakerDAO}\footnote{\url{https://makerdao.com/}}

It's the protocol behind the stable coin Dai -- a cryptocurrency that maintains a 1:1 peg to the USD, being backed by collateral (Ether to be specific).

Let's say you're an Ether holder and you would like to create Dai. Your first move would be to send your Ether to a \textbf{Collateralized Debt Position (CDP)}. A CDP is a type of software that runs on the blockchain, in this case the Ethereum blockchain, and lives within the Maker ecosystem.

Once your Ether is in the CDP smart contract, you are able to create Dai. The amount of Dai you can create is relative to how much Ether you have put into the CDP. This ratio is fixed, but can be changed over time. The amount of Dai you can create relative to the Ether you put in is called the \textbf{collateralization ratio}.

Assume 1 Ether = \$100 and collateralization ratio is 1.5. If you send 1 Ether into the CDP smart contract, then you are now able to create 66 Dai, i.e. 100/1.5.

\begin{itemize}
	\item \textbf{what happens when Ether goes up?}\\
	The system becomes over collateralized and Dai becomes stronger.
	Maker has mechanisms that incentivize users to create more Dai if the price of Dai should trade above one dollar.
	\item \textbf{what happens when Ether goes down?}\\
	If Ether goes down, now that can cause problems.\\
	If the value of Ether held as collateral is worth less than the amount of Dai it's supposed to be backing, then Dai would not be worth one dollar and the system could collapse.\\
	Maker combats this by liquidating CDPs and auctioning off the Ether inside before the value of the Ether is less than the amount of Dai it is backing.\\
	Basically, if the price feed into the CDP indicates that the value of Ether has gone below a certain threshold, the liquidation ratio (let's use 125\% of created Dai), then the CDP is ``liquidated'' and the Ether inside the CDP is auctioned off for Dai until there is enough Dai to pay back what was extracted from the CDP.
\end{itemize}

\textbf{Volatility cannot be destroyed}, it can only be transferred. If we have a stable token like Dai that has been stripped of its volatility, where did it go? In the Maker system, volatility is transferred entirely to the holder of the CDP. Using our prior example, should I withdraw 66 Dai from a CDP containing one ether, I will only own that one Ether if its price is above the liquidation ratio. Dai is effectively a loan on my ether.

This leads us to the interesting consequence: I can take the Dai that I borrowed and use it to buy more ether. By doing this, I am basically buying Ether on margin. That's right, completely decentralized leverage!

MakerDAO is like a credit facility that issues loans with a certain interest rate. If the interest rate (stability fee) is low, people are encouraged to borrow more (lock up more ETH). If the interest rate is high, the cost of capital is high making it less attractive to borrow (close out CDPs).

Tether is pegged to an asset, MakerDAO is tied to the amount you put in a CDP and can cause problems because Ether is much more volatile.

\fg{1}{bdcc146817b36f24613f0c0967a7d1f9-1ruH8716er-image}

\paragraph{Algorithmic Stablecoins}\leavevmode
\url{https://www.youtube.com/watch?v=S7-rfvpEpJs}

\fg{1}{7dc3967046522e1c02d9c1916fa87f9b-DMNNfblLZC-image}

\fg{1}{49d147896a4ef0b2ca9045ac755abe65-Sx4PA1dE8U-image}

\section{Cryptocurrencies and Asset Allocation}

Some meaningful questions:
\begin{itemize}
	\item Are expected returns higher than (classical) financial assets ones?
	\item Are cryptocurrencies riskier than (classical) financial assets?
	\item What about diversification?
\end{itemize}

Let $\pi =(w_{1},\ldots,w_{N})$ be a portfolio, where $w_{i}$ is the amount of asset $i$, and is positive if bought, negative if short.

\paragraph{Mean-Variance Analysis}

It's the process of weigthing risk, expressed as variance, against expected return, \textbf{applied to a portfolio, not a single asset}.

Investors use mean-variance analysis to make decisions about which financial instruments to invest in, based on how much risk they are willing to take on in exchange for different levels of reward.

The \textbf{efficient frontier} is the set of optimal portfolios that offer the highest expected return for a defined level of risk or the lowest risk for a given level of expected return.

\fg{0.5}{efficient-frontier}

$N$ assets: $R=(\bar{r}_1,\ldots,\bar{r}_N)$ is the vector of expected return. $V$ is the covariance matrix.

\[
	\bar{R} = \pi R \qquad \bar{\sigma}^2 = \pi V \pi \transpose
\]

are the expected return and variance of the portfolio.

\paragraph{Our market for the first (toy) problem}

\begin{itemize}
	\item Bitcoin
	\item Ether
	\item Amazon
	\item Apple
	\item Facebook
	\item General Electric
	\item General Motor
\end{itemize}

Daily datas (only working day) from September 2017 to February 2020

\begin{gather*}
\begin{bmatrix}
\text{return}\\
\text{variance}
\end{bmatrix} =\begin{bmatrix}
1.0026 & 1.0017 & 1.0014 & 1.0012 & 1.0005 & 0.9993 & 1.0000\\
0.24 & 0.36 & 0.03 & 0.03 & 0.04 & 0.06 & 0.03
\end{bmatrix}\\
\\
\text{correlation} =\begin{bmatrix}
1.0000 & 0.6975 & 0.0149 & 0.0188 & 0.0350 & 0.0160 & -0.0093\\
0.6975 & 1.0000 & 0.0546 & 0.0614 & 0.0441 & 0.0403 & 0.0561\\
0.0149 & 0.0546 & 1.0000 & 0.5807 & 0.5623 & 0.2261 & 0.3016\\
0.0188 & 0.0614 & 0.5807 & 1.0000 & 0.4706 & 0.2844 & 0.3454\\
0.0350 & 0.0441 & 0.5623 & 0.4706 & 1.0000 & 0.2132 & 0.3292\\
0.0160 & 0.0403 & 0.2261 & 0.2844 & 0.2132 & 1.0000 & 0.2809\\
-0.0093 & 0.0561 & 0.3016 & 0.3454 & 0.3292 & 0.2809 & 1.0000
\end{bmatrix}
\end{gather*}

\textbf{Portfolio 1}
\[
	\pi = \begin{bmatrix}
	0.9573 & -0.2611 &  2.8507 &  2.7995 & -1.4778 & -2.1661 & -1.7026
	\end{bmatrix}
	\qquad
	\bar{R} = 1.01
\]

\textbf{Portfolio 2}
\[
	\pi = \begin{bmatrix}
	0.1037 & -0.0223 &  0.2308 &  0.3644 &  0.0439 &  0.0252 &  0.2543
	\end{bmatrix}
	\qquad
	\bar{R} = 1.001
\]

\textbf{Portfolio 3}
\[
	\pi = \begin{bmatrix}
	0.0184 &  0.0015 & -0.0312 &  0.1209 &  0.1961 &  0.2444 &  0.4500
	\end{bmatrix}
	\qquad
	\bar{R} = 1.0001
\]

Suppose to invest \$100. After one year:
\begin{itemize}
	\item Portfolio 1: $100 \cdot 1.01^{365} = 3778$
	\item Portfolio 2: $100 \cdot 1.001^{365} = 144$
	\item Portfolio 3: $100 \cdot 1.0001^{365} = 103$
\end{itemize}

The best reason to invest in cryptocurrency seems \textbf{diversification}.

\paragraph{Our market for the second (toy) problem}

\begin{itemize}
	\item First 50 stocks for capitalization (S\&P50)
	\item 4 cryptocurrencies: Bitcoin, Ethereum, Ripple, Litecoin
\end{itemize}

Period: July 2015-May 2019

\fg{0.7}{a091830eaba227339add9fbe000a7964-Xjp2TOr5Gl-image}

Cryptocurrencies are very uncorrelated with standard financial assets, as can be proved using statistics. However, they are quite correlated among them.

\fg[Correlation.]{0.7}{a9628e77047a0bb29a84c79a86340735-oczvbgskgs-image}

\fg[Mean-Variance Analysis]{0.7}{toy2}

\section{Central Bank Digital Currency (CBDC)}

Central bank money handled through electronic means and accessible to the broad public. Responsibility for the Central Bank.

CBDC is a third form of base money, next to deposits and banknotes.

The four key properties of Money:
\begin{itemize}
	\item Issuer: central bank or not
	\item Form: digital or physical
	\item Accessibility: widely or restricted
	\item Technology: token - or account - based. The key difference between tokens and accounts is in their verification: a person receiving a token verifies that token is genuine, an intermediary verifies the identity of an account holder.
\end{itemize}


\textbf{Cash features}
\begin{itemize}
	\item Inclusive: easy to use and available to everybody without electronic devices.
	\item Crisis proof: device independent (power blackout).
	Peer to peer transactions: there is no intermediation of a thrid party to complete a transaction.
	\item Anonymous.
	\item Off-line transactions.
\end{itemize}

Digital means of payments do not satisfy these features. Digital means require third parties (record keeping, identity validation). It si difficult to deploy off-line transactions.
Today: cash is the only form of central bank issued money available to individuals.
Today: part of money (bank deposits) is managed by profit oriented players.
CBDC: universal/inclusive, digital, central bank issued (legal tender)


% vantaggi del cash sul digitale

2. State of the art

3. Motivation

The Consumer perspective
The Central Bank perspective
The real economy perspective

% robe varie che non ho segnato





















































































































\chapter{Blockchain (N. C. Fabrizio)}

\section{Intro to DTL}

Bitcoin is the mother of all the blockchains. It's constructed around the idea of having a \textbf{ledger} where we track information. Typically there is a \textbf{central} authority which holds the only copy of it and is the only one allowed to modify it.

A distributed ledger is a \textbf{tamperproof} sequence of data that can be \textbf{read} and \textbf{augmented} by \textbf{everyone}. Nodes don't need to \textbf{trust} each other, the mechanism of \textbf{consensus} (a combination of game theory and probability) removes the need for it, and everyone has an interest in behaving as the blockchain requires.

DLT: which desirable properties?
\begin{itemize}
	\item Immutability
	\item Accessibility
	\item Decentralisation
\end{itemize}

\section{Bitcoin Protocol}

Blockchain: a data structure to record information incrementally (a history). An append-only data structure, a growing chain of blocks, with immutable history: what is written cannot be altered.

It's maintained by a \textbf{peer-to-peer} open network of nodes: anyone can join! Identical replicated copies of the ledger/blockchain, \textbf{openly available}: anyone can read, \textbf{true by majority}.

PoW slows and controls the creation of new blocks, in this way you can't change the blockchain previously and have the time to re-calculate all hashes, because the chain will grow and the longest chain always wins (except with hard-forks).

No one is in charge, but the whole community.

Problem: how does the community select someone in charge to write the next block?

Who writes the next sentence?

In our case, whoever writes the next sentence gets \numprint{100000} EUR!

Who writes the next sentence?

Proof of Work! This is the solution adopted by the Bitcoin blockchain: by controlling the difficulty of the challenge (against the available computational power in the network), it is possible to select, with a reasonable probability, a leader in charge of the next sentence/block. Note that the challenge is difficult to solve, i.e. find 431 and 433, but easy to validate, i.e. multiply them and check the result: difficult to be elected, straightforward to be recognized as a leader.

\paragraph*{Bitcoin mining protocol [Nakamoto08]}
Each node of the Bitcoin network will
\begin{itemize}
	\item choose and verify pending transactions
	\item solve the crypto-puzzle (depending on the previous block)
\end{itemize}

If solved/leader then
\begin{itemize}
	\item create the next block and
	\item broadcast the new block
\end{itemize}

Everyone
\begin{itemize}
	\item validate new block (Proof of Work + transactions) and embed it in the local copy of the blockchain
	\item start mining the next block
\end{itemize}

The blockchain is actually a tree, including \emph{the} chain.

Currently, a block is about 1 MB and is added every 10 minutes (for an average of 3-7 trans/sec).

If a node wants to change an old block, or in general wants to make the chain take a direction he wants, he would need half as much the computational power of the network to keep up with the others. This is considered to be impossible, and everyone has an advantage in sticking with the same version.

\section{Security aspects}

\subsection{Hash functions}

\paragraph*{Hashes.} Are functions: $H(a)=h$ such that:
\begin{itemize}
	\item from $a$ to $h$ is easy, but from $h$ to $a$ computationally infeasible (you should try every possible input);
	\item $a$ can be of any length (thus could even be a document) but $h$ is fixed by the specific hash function chosen.
\end{itemize}

\textbf{Properties:}
\begin{itemize}
	\item \textbf{Collision resistance}, it's infeasible to find $a$ and $b$ such that $a\neq b$ and $H(a)=H(b)$
	\item \textbf{Hiding}, when a secret value $r$ is chosen from a probability distribution that has high entropy, then given $H(r\mid x)$ it is infeasible to find $x$. ``$\mid$'' means concatenation of two strings.
	\item \textbf{Puzzle friendliness}, given $h$ and a random $k$, it's unfeasible to find $x$ such that $h=H(k\mid x)$.
\end{itemize}

\subsection{Merkle tree}

Some desirable properties of a (new) block
\begin{itemize}
	\item follows linearly (and exclusively) the previous one - links to the previous block
	\item the set of transactions is fixed before leader election - Merkle tree. against corruption
	\item the leader is identified – as the solver of the puzzle
\end{itemize}

\paragraph*{Merkle tree.} Is a digest of digests, i.e. a tree, a compact representation of a set of hashes of Bitcoin transactions, amongst the proposed and selected ones. The root represents all the hashes in the tree and any change anywhere would affect the root.
\begin{itemize}
	\item Transactions are hashed two by two until the root, which is like the DNA of the block.
	\item In other words, a Merkle tree summarizes all the transactions in a block by producing a digital fingerprint of the entire set of transactions, thereby enabling a user to verify whether or not a transaction is included in a block.
	\item It is possible to \textbf{verify that a transaction belongs to a block without verifying all the blocks.}
	\item Thanks to this structure, it is quicker and more efficient to verify the consistency and content of the data
\end{itemize}

\fg{0.7}{a4e440abba339ed859d31f59f5a84ad6-6M8F4XuJr8-image}

\subsection{Proof of Work}

\paragraph*{Proof of Work} (at a certain degree of approximation) is finding \texttt{nonce} (number only used once) such that
\begin{equation*}
	H(H(\text{previous block}),M(\text{transactions} +\text{revenue transaction}),\texttt{nonce})
\end{equation*}
is an hash starting with $C$ zeros. $M$ is the Merkle Root of the input transaction and the reward transaction. $C$ controls the complexity and adapts regularly. There are some technical details involved, like the fact the timestamps are also hashed, everything has to be converted to hexadecimal and the hash is made two times.

PoW is very expensive in terms of energy consumption.

Miners have interest in solving this hard problem because they are rewarded:
\begin{itemize}
	\item the fees from each transaction;
	\item an amount of BTC generated (mined), in other words, everyone suddenly agrees that such miner has that quantity in addition to their wallet. The quantity was initially 50 BTC and is halved every \numprint{210000} blocks; currently, it's 6.25 BTC. This mechanisms guarantees that in circulation there will never be more than
	\begin{equation*}
		\numprint{210000} \cdot (50+25+12.5+6.25+\cdots) = \numprint{21000000}\ \text{BTC}
	\end{equation*}
\end{itemize}

\subsection{Public and private keys}

Asymmetric\footnote{\textit{asymmetric} means that the two keys are different.} cryptography is based on two keys associated with an identity:
\begin{itemize}
	\item a private (secret) key $sk$, known only by the owner. It is generally used to encrypt a message by the owner, $sk\{m\}$
	\item a public key $pk$ associated to the identity and public. It can decrypt a message encrypted with $sk$, i.e. $pk\{sk\{m\}\} =m$ and also $sk\{pk\{m]\} =m$.
\end{itemize}

\textbf{Properties}
\begin{itemize}
	\item one way function: unfeasible from $sk\{m\}$ to get $m$ without $pk$, easy with $pk$.
	\item with $n$ nodes, only $n$ pairs ($pk_{i}$,$sk_{i}$) are needed instead of $n^{2}$
	\item anyone can secretly send to $j$ by using $pk_{j}\{m\}$
	\item $j$ can prove that they own a piece of information by using $sk_{j}\{m\}$
\end{itemize}

Signature of a message (of a transaction) – on top of asymmetric cryptography:
\begin{itemize}
	\item $sig:=sign(sk,\text{message})$
	\item $isValid:=verify(pk,\text{message},sig)$
\end{itemize}

Using the public key $pk$ one can validate the author of a message (transaction!)

How is asymmetric cryptography is used in Bitcoin?

You generate your own private key from a very large set, from there you derive your public key, and from the public key, an address can be derived by hashing.

\fg{0.7}{f468dea48fcab290a0d662d26e78de02-t8qu0CuuVm-image}

Bitcoin transactions move value from multiple input addresses to multiple output addresses.

\subsection{Encryption with Elliptic Curves}

In algebraic geometric, an elliptic curves (EC) over $\mathbb{R}$ is defined by the (Weirstrass's equation): $y^{2} =x^{3} +ax+b$. The curve is non singular if its determinant is non zero, which means $4a^{3} +27b^{2} \neq 0$.

Bitcoin uses EC for public key, it uses a famous EC \textbf{secp256k1}:
\begin{equation*}
	y^{2} =x^{3} +7
\end{equation*}

\url{https://youtu.be/muIv8I6v1aE}

\url{https://youtu.be/qCafMW4OG7s}

Bitcoin uses elliptic curves for public keys.

\section{Structure of a Block}

A block has two main components:
\begin{itemize}
	\item \textbf{list of transactions};
	\item \textbf{block header}, which contains:
	\begin{itemize}
		\item \texttt{Version}: Block version number
		\item \texttt{Previous Block hash}: This is used to compute a new block hash. Hence, making the blockchain temper-proof.\footnote{The first block doesn't have any previous one, it's called the \textbf{genesis block}.}
		\item \texttt{Merkle Tree Root}: The root of the Merkle tree is a verification of all the transactions.
		\item \texttt{Timestamp}: The time at which block is mined.
		\item \texttt{Bits/Difficulty}: Difficulty in Bitcoin is expressed by the hash of a Bitcoin block header being required to be numerically lower than a certain target.
		\item \texttt{Nonce}: A 32-bit random number used in blockchain, while calculating the cryptographic hash for a Block.
	\end{itemize}
\end{itemize}

\section{Public, Hybrid and Private DLT}

The ledger, replicated on each node, may be \textbf{public}, \textbf{private} or \textbf{hybrid}.

The network maybe \textbf{permissionless} or \textbf{permissioned}.

\textbf{Public}
\begin{itemize}
	\item No one is in charge
	\item Anyone can participate
	\item Open and transparent
	\item Decentralized consensus mechanisms (PoW, PoS, ...)
	\item Mining
	\item Slow
\end{itemize}

\textbf{Hybrid}
\begin{itemize}
	\item Multiple selected organizations
	\item Permissioned known identities
	\item Pre-approved participants
	\item Voting/multi- party consensus
	\item Lighter and faster
\end{itemize}

\textbf{Private}
\begin{itemize}
	\item Private property of an
	\item individual or an organization
	\item Permissioned known identities
	\item Pre-approved participants
	\item Voting/multi- party consensus
	\item Lighter and faster
\end{itemize}

\fg{1}{0ab69fc7ee15e9b5efb04bb8d39299ab-2lIIJXa4gj-image}

In general, every blockchain has its token.

\section{Ethereum}

Ethereum is an \textbf{open-source} and public blockchain-based \textbf{distributed computing platform for building decentralized applications}. Vitalik Buterin envisioned Ethereum as a platform for developers to write programs on the blockchain. In short, Ethereum wants to be a \textbf{World Computer} that would decentralize – and some would argue, democratize – the existing client-server model. With Ethereum, servers and clouds are replaced by thousands of so-called ``nodes'' run by volunteers from across the globe (thus forming a ``world computer'').

\textbf{Smart contracts} are simple programs stored on the blockchain that automatically exchange money based on certain conditions.

A contract that self-executes, and handles the enforcement, the management, performance, \& payment. You would require tokens for executing a smart contract as well as for trading. So basically, Ethereum is incomplete without cryptocurrency. The language is \textit{Turing-complete}, meaning it supports a broader set of computational instructions.

Smart contracts can:
\begin{itemize}
	\item Function as \textit{multi-signature} accounts, so that funds are spent only when a required percentage of people agree
	\item Manage agreements between users, say if one buys insurance from the other
	\item Provide utility to other contracts (similar to how a software library works)
	\item Store information about an application, such as domain registration information or membership records.
\end{itemize}

Ethereum runs on its native token (\textbf{Ether}) which serves two main purposes:
\begin{itemize}
	\item Ether payment is \textbf{required for applications to perform any operation} so that broken and malicious programs are kept under control.
	\item Ether is \textbf{rewarded as an incentive to the miners who contribute to the Ethereum network} with their resources - much like Bitcoin's structure.
\end{itemize}

Every time a contract is executed, Ethereum consumes a token which is termed as \textbf{gas} to run the computations. Gas is required to be paid for every operation performed on the Ethereum blockchain. Its price is expressed in Ether and it's decided by the miners, which can refuse to process the transaction with less than a certain gas price.

\textbf{Examples}\footnote{\url{https://medium.com/coreledger/what-are-smart-contracts-a-breakdown-for-beginners-92ac68ebdbeb}}
\begin{itemize}
	\item A smart contract could be programmed to release funds for someone's birthday each year. It could also be programmed to release payment once someone confirms receipt of delivered goods. It could be used to enforce particular rights for holders of digital assets.
	\item A simple example could be in the case of \textbf{life insurance}. The policy terms would be encoded into the smart contract. In the event of a passing, the notarized death certificate would be provided as the input trigger for the smart contract to release the payment to the named beneficiaries.
\end{itemize}

\paragraph{Oracle} A blockchain oracle is a third-party service that connects smart contracts with the outside world, primarily to feed information in from the world, but also the reverse. Information from the world encapsulates multiple sources so that decentralised knowledge is obtained. Information to the world includes making payments and notifying parties. The oracle is the layer that queries, verifies, and authenticates external data sources, usually via trusted APIs, proprietary corporate data feeds and internet of things feeds and then relays that information.

\section{Ethereum vs. Bitcoin}

While both the Bitcoin and Ethereum networks are powered by the principle of distributed ledgers and cryptography, the two differ technically in many ways. For example, \textbf{transactions on the Ethereum network may contain executable code, while data affixed to Bitcoin network transactions are generally only for keeping notes.}

Both Bitcoin and Ethereum currently use a consensus protocol called Proof of Work (PoW), which allows the nodes of the respective networks to agree on the state of all information recorded on their blockchains and prevent certain types of economic attacks on the networks. In 2022, \textbf{Ethereum will be moving to a different system called Proof of Stake (PoS)} as part of its Eth2 upgrade, a set of interconnected upgrades that will make Ethereum more scalable, secure, and \textbf{sustainable}.

Proof of Stake substitutes computational power with staking-making it less energy-intensive-and replaces miners with validators, who stake their cryptocurrency holdings to activate the ability to create new blocks.

While Bitcoin was created as an alternative to national currencies and thus aspires to be a medium of exchange and a store of value, \textbf{Ethereum was intended as a platform to facilitate immutable, programmatic contracts and applications via its own currency}.

\section{Ethereum moving to Proof of Stake}

Ethereum has announced the launch of 2.0 and the process of moving to Proof

of Stake (may be completed and happen soon)
\begin{itemize}
	\item This approach does not require large expenditures on computing and energy, it will change the need for CPUs for mining
	\item Miners are now ``validators'' and post a deposit in an escrow account
	\item The more escrow you post, the higher the probability you will be chosen to nominate the next block
	\item If you nominate a block with invalid transactions, you lose your escrow
	\item One issue with this approach is that those that have the most Ethereum will be able to get even more
	\item This leads to centralization eventually
	\item On the other hand, it reduces the chance of a 51\% attack and allows for near-instant transaction approvals
	\item The protocol is called Casper and this will be a hard fork
\end{itemize}

\section{Dapps}

\begin{itemize}
	\item \textbf{Decentralized Applications (Dapps)} are computer applications that operate over a blockchain enabling direct interaction between end-users and providers.
	\item \textbf{The interface of the decentralized applications does not look any different than any website or mobile app today.}
	\item The \textbf{smart contract represents the core logic} of a decentralized application. Smart contracts are integral building blocks of blockchains, that process information from external sensors or events and help the blockchain manage the state of all network actors.
\end{itemize}

\section{ERC20}

\begin{itemize}
	\item First token standard
	\item Introduced in November 2015 as an Ethereum Request for Comments (ERC)
	\item Automatically assigned GitHub issue number 20, giving rise to the name ``ERC20''
	\item A standard for fungible tokens, meaning that different units of an ERC20 token are interchangeable and have no unique properties
	\item The ERC20 protocol standard contains basic functions that any useful token should implement to enable trading. These include transferring tokens, inquiring about the balance of tokens at a certain address, and the total supply of tokens.
	\item The ERC-20 standard defines the interfaces for a few common methods: i.e. totalSupply, balanceOf, transfer, transferFrom, and approve. These methods allow Ethereum smart contracts to issue fungible tokens and token holders to transfer tokens to one another.\footnote{\url{https://eips.ethereum.org/erc}}
	\item Today, there are thousands and thousands of ERC20 tokens (derivatives of ETH), almost in every sector and use case, not only in finance (for a list see Etherscan and select ERC20)
\end{itemize}
